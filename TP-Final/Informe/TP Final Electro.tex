\documentclass[a4paper]{article}
\usepackage[utf8]{inputenc}
\usepackage[spanish, es-tabla]{babel}

\usepackage{amsmath}
\usepackage{amsfonts}
\usepackage{amssymb}

\usepackage{float}
\usepackage{graphicx}
\graphicspath{ {./Imagenes/} }

\usepackage{multirow}
\setlength{\doublerulesep}{\arrayrulewidth}

\usepackage{array}
\newcolumntype{C}[1]{>{\centering\let\newline\\\arraybackslash\hspace{0pt}}m{#1}}

\usepackage[american]{circuitikz}

\usepackage{fancyhdr}

\usepackage{units} 

\pagestyle{fancy}
\fancyhf{}
\lhead{22.02 Electrotecnia I}
\rhead{Mechoulam, Mestanza, Lambertucci, Pouthier, Londero}
\rfoot{Página \thepage}



\begin{document}

%%%%%%%%%%%%%%%%%%%%%%%%%%%%%%%%%%%%%%%%%%%%%%%%%%%%%%%%%%%%%%%%%%%%%%%%% 
%								CARATULA								%
%%%%%%%%%%%%%%%%%%%%%%%%%%%%%%%%%%%%%%%%%%%%%%%%%%%%%%%%%%%%%%%%%%%%%%%%% 

\begin{titlepage}
\newcommand{\HRule}{\rule{\linewidth}{0.5mm}}
\center
\mbox{\textsc{\LARGE \bfseries {Instituto Tecnológico de Buenos Aires}}}\\[1.5cm]
\textsc{\Large 22.02 Electrotecnia I}\\[0.5cm]


\HRule \\[0.6cm]
{ \Huge \bfseries Trabajo Práctico Final}\\[0.4cm] 
\HRule \\[1.5cm]


{\large

\emph{Grupo 5}\\
\vspace{3px}

\begin{tabular}{lr} 	
\textsc{Mechoulam}, Alan  &  58438\\
\textsc{Lambertucci}, Guido Enrique  & 58009 \\
\textsc{Pouthier}, Florian  & 61337 \\
\textsc{Mestanza}, Nicolás  & 57521 \\
\textsc{Londero Bonaparte}, Tomás Guillermo  & 58150 \\
\end{tabular}

\vspace{20px}

\emph{Profesores}\\
\vspace{3px}
\textsc{Muñoz}, Claudio Marcelo\\ 	
\textsc{Ayub}, Gustavo\\ 	

\vspace{100px}

\begin{tabular}{ll}

Presentado: & ??/06/19\\

\end{tabular}

}

\vfill

\end{titlepage}


%%%%%%%%%%%%%%%%%%%%%%%%%%%%%%%%%%%%%%%%%%%%%%%%%%%%%%%%%%%%%%%%%%%%%%%%% 
%								INFORME									%
%%%%%%%%%%%%%%%%%%%%%%%%%%%%%%%%%%%%%%%%%%%%%%%%%%%%%%%%%%%%%%%%%%%%%%%%%

\section*{Introducción}

En el siguiente informe se expone un resumen el trabajo realizado en \textbf{Altium}, \textbf{LTSpice} y \textbf{Python}. En este expone los resultados obtenidos de las simulaciones realizadas.

\section*{Desarrollo}

\subsection*{Modelado de placa en Altium}

\subsection*{Simulación en LTSpice}

Se comenzó analizando el período de carga y descarga de los componentes capacitivos e inductivos del circuito dado.

\begin{figure}[H]
\begin{center}
\begin{circuitikz}
\draw
	(0,3) to [V,v_=$V$] (0,0)
	(0,0) to (4,0)
	(2,0) to [R, l_=$18 \ \Omega$] (2,1.5)
	(4,0) to [R, l_=$12 \ \Omega$] (4,1.5)
	(2,1.5) to [C, l_=$22 \ \mu f$]	(2,3)
	(4,3)	to [L, l=$13 \ mH$] (4,1.5)
	(0,3)	to [spst] (2,3)
	(2,3)	to (4,3)
	
;\end{circuitikz}
\caption{Circuito analizado.}
\end{center}
\end{figure}

Mediante el uso de \textbf{LTSpice}, y otorgando un valor arbitrario de $5 \ V$ a la tensión de entrada, se obtuvo

\begin{figure}[H]
	\centering
	\includegraphics[width=\textwidth]{LTSpice-Carga1}
	\caption{Tensión del capacitor (en rojo) y corritente en la bobina (en azul) durante la carga.}
	\label{fig:LTSC1}
\end{figure}

\begin{figure}[H]
	\centering
	\includegraphics[width=\textwidth]{LTSpice-Descarga1}
	\caption{Tensión del capacitor (en rojo) y corritente en la bobina (en azul) durante la descarga.}
	\label{fig:LTSD1}
\end{figure}

Mediante el uso de las figuras (\ref{fig:LTSC1}) y (\ref{fig:LTSD1}) se pudo determinar la pseudofrecuencia de oscilación del transitorio y el valor máximo de sobrepico. De esta forma, se determinó para el capacitor una pseudofrecuencia de ${\omega}_{T} \ = \ 4.47 \ ms$ y un sobrepico de $227 \ mV$, y para el inductor una pseudofrecuencia de ${\omega}_{T} \ = \ 3.84 \ ms$ y un sobrepico de $4.55 \ mA$.

Para analizar teóricamente ambas situaciones se observa que tanto el capacitor como el inductor se encuentran en ramas distintas conectadas a una fuente de tensión constante, es por ello que se llega a las siguientes ecuaciones:

\begin{equation}
	V_{C} (t) = V_{f} \left( 1 - e^{- \frac{t}{0.396 \ ms}  } \right)
	\label{eq:carg-c}
\end{equation}

\begin{equation}
	I_{L} (t) = \frac{V_{f}}{30 \ \Omega} \left( 1 - e^{- \frac{t}{0.156 \ s}  } \right)
	\label{eq:carg-l}
\end{equation}

%Por otro lado, para la descarga, el circuito es un RLC serie. Sabiendo que para este circuito $\alpha < \omega_0$, se plantea la solución para un circuito subamortiguado, obteniéndose así las siguientes ecuaciones:
%
%\begin{equation}
%	V_{C} (t) = V_{f} \cdot e^{- \alpha t} \left[ cos \left(  \omega_d t \right) + {\omega_d}^{-1}  \left( \alpha + \frac{R}{C} \right) sen \left(  \omega_d t \right) \right]
%	\label{eq:descarg-c}
%\end{equation}
%
%\begin{equation} \label{eq:descarg-l}
%\begin{split}
%I_{L} (t) = V_{f} C \cdot e^{- \alpha t} \left\lbrace - \alpha \left[ cos \left(  \omega_d t \right) + {\omega_d}^{-1}  \left( \alpha + \frac{R}{C} \right) sen \left(  \omega_d t \right) \right] \right. \\
% \left. - \omega_d sen \left(  \omega_d t \right) + \left( \alpha + \frac{R}{C} \right) cos \left(  \omega_d t \right) \right\rbrace
%\end{split}
%\end{equation}
%
%con $\alpha = 1153.85 \ \frac{1}{s}$, $\omega_d = 1471.44 \ \frac{1}{s}$, $R = 30 \ \Omega$.

Luego se analiza el caso en el cual el valor de ambas resistencias sea de $0 \ \Omega$. Idealmente, lo que sucedería sería que la tensión y la corriente oscilarían entre el capacitor y la bobina sin perdidas de energía. Dicha situación no es posible en la realidad ya que siempre se presenta una resistencia interna por parte de los elementos, lo que generaría perdidas. 

Finalmente, se realizó un diagrama de Montecarlo. Para ello se determinó las tolerancias de los distintos componentes de la siguiente forma: 5\% para las resistencias, 10\% para el capacitor y 0\% para la bobina. Se corrió la simulación 100 veces con un punto inicial en 1 e incrementándolo en 1 por iteratición. Además, se configuró la fuente de tensión de entrada para que varíe desde $0 \ V$ a $5 \ V$ con un paso de $0,1 \ V$. De esta forma se obtuvieron los gráficos presentados en la figuras (\ref{fig:LTSMCVC}), (\ref{fig:LTSMCIL}) y (\ref{fig:LTSMCIR}).

\begin{figure}[H]
	\centering
	\includegraphics[width=\textwidth]{LTSpice-MC1-VC}
	\caption{Análisis de Montecarlo de la tensión en el capacitor.}
	\label{fig:LTSMCVC}
\end{figure}

\begin{figure}[H]
	\centering
	\includegraphics[width=\textwidth]{LTSpice-MC1-IR}
	\caption{Análisis de Montecarlo de la corriente en la resistencia 1.}
	\label{fig:LTSMCIR}
\end{figure}

\begin{figure}[H]
	\centering
	\includegraphics[width=\textwidth]{LTSpice-MC1-IL}
	\caption{Análisis de Montecarlo de la corriente en la bobina (igual corriente que en la resistencia 2).}
	\label{fig:LTSMCIL}
\end{figure}

\subsection*{GUI en Python}

\section*{Conclusión}

Se pudo predecir los resultados obtenidos en las gráficas de corriente en función de la tensión en base a los conocimientos teóricos. Por otro lado, se pudo observar la importancia de las tolerancias y como afectan estas al resultado original mediante el uso del análisis de Montecarlo. Ademas se destacan dos puntos, el primero es como las tolerancias de los dispositivos afectan a los demás, como es el caso de la resistencia 2 a la corriente de la bobina, mientras que el segundo es la poca variación en el capacitor. 

\end{document}
