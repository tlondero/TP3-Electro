\documentclass[a4paper]{article}
\usepackage[utf8]{inputenc}
\usepackage[spanish, es-tabla]{babel}

\usepackage{amsmath}
\usepackage{amsfonts}
\usepackage{amssymb}

\usepackage{float}
\usepackage{graphicx}
\graphicspath{ {./Imagenes/} }

\usepackage[american voltage]{circuitikz}

\usepackage{fancyhdr}

\usepackage{units} 

\pagestyle{fancy}
\fancyhf{}
\lhead{22.02 Electrotecnia I}
\rhead{Mechoulam, Mestanza, Lambertucci, Pouthier, Londero}
\rfoot{Página \thepage}



\begin{document}

%%%%%%%%%%%%%%%%%%%%%%%%%%%%%%%%%%%%%%%%%%%%%%%%%%%%%%%%%%%%%%%%%%%%%%%%% 
%								CARATULA								%
%%%%%%%%%%%%%%%%%%%%%%%%%%%%%%%%%%%%%%%%%%%%%%%%%%%%%%%%%%%%%%%%%%%%%%%%% 

\begin{titlepage}
\newcommand{\HRule}{\rule{\linewidth}{0.5mm}}
\center
\mbox{\textsc{\LARGE \bfseries {Instituto Tecnológico de Buenos Aires}}}\\[1.5cm]
\textsc{\Large 22.02 Electrotecnia I}\\[0.5cm]


\HRule \\[0.6cm]
{ \Huge \bfseries Trabajo Práctico Final}\\[0.4cm] 
\HRule \\[1.5cm]


{\large

\emph{Grupo 5}\\
\vspace{3px}

\begin{tabular}{lr} 	
\textsc{Mechoulam}, Alan  &  58438\\
\textsc{Lambertucci}, Guido Enrique  & 58009 \\
\textsc{Pouthier}, Florian  & 61337 \\
\textsc{Mestanza}, Nicolás  & 57521 \\
\textsc{Londero Bonaparte}, Tomás Guillermo  & 58150 \\
\end{tabular}

\vspace{20px}

\emph{Profesores}\\
\vspace{3px}
\textsc{Muñoz}, Claudio Marcelo\\ 	
\textsc{Ayub}, Gustavo\\ 	

\vspace{100px}

\begin{tabular}{ll}

Presentado: & ??/06/19\\

\end{tabular}

}

\vfill

\end{titlepage}


%%%%%%%%%%%%%%%%%%%%%%%%%%%%%%%%%%%%%%%%%%%%%%%%%%%%%%%%%%%%%%%%%%%%%%%%% 
%								INFORME									%
%%%%%%%%%%%%%%%%%%%%%%%%%%%%%%%%%%%%%%%%%%%%%%%%%%%%%%%%%%%%%%%%%%%%%%%%%

\section*{Introducción}

En la experiencia realizada se buscó analizar las potencias reactivas, activas y aparentes en los distintos tipos de circuitos propuestos.

\section*{Desarrollo de la experiencia}

\subsection*{\underline{Ejercicio 1}}

En esta etapa, se analizó un circuito con una bobina, la cual dispone de un núcleo de hierro macizo, lo que permitió observar tres situaciones distintas:
\begin{enumerate}
	\item[A)] La totalidad del núcleo dentro de la bobina.
	\item[B)] La mitad del núcleo dentro de la bobina.
	\item[C)] Sin núcleo.
\end{enumerate}

\begin{figure}[H]
	\centering
	\includegraphics[width=0.6\textwidth]{Circuito-ejercicio-1A}
	\caption{Circuito con la totalidad del núcleo dentro de la bobina.}
	\label{fig:1a}
\end{figure}

\begin{figure}[H]
	\centering
	\includegraphics[width=0.6\textwidth]{Circuito-ejercicio-1B}
	\caption{Circuito con la mitad del núcleo dentro de la bobina.}
	\label{fig:1b}
\end{figure}

\begin{figure}[H]
	\centering
	\includegraphics[width=0.6\textwidth]{Circuito-ejercicio-1C}
	\caption{Circuito con sin núcleo.}
	\label{fig:1c}
\end{figure}

De las mediciones realizadas, se pudo confeccionar la siguiente tabla:

\begin{table}[H]
\centering
\begin{tabular}{|c|c|c|c|c|c|c|}
\hline
\textbf{Circuito}                                                       & \textbf{P {[}W{]}} & \textbf{I {[}A{]}} & \textbf{V {[}V{]}} & \textbf{Q {[}VAR{]}} & \textbf{S {[}VA{]}} & \textbf{Cos($\varphi$)} \\ \hline
A) Núcleo sólido                                                        & 7,25               & 0,28               & 99,50              & 26,38                & 27,36               & 0,26                                \\ \hline
\begin{tabular}[c]{@{}c@{}}B) Núcleo sólido\\ por la mitad\end{tabular} & 9,88               & 0,46               & 98,50              & 43,72                & 44,82               & 0,22                                \\ \hline
C) Sin núcleo                                                           & 23,25              & 0,92               & 96,00              & 85,40                & 88,51               & 0,26                                \\ \hline
\end{tabular}
\caption{Tesiones, corrientes y potencias medidas y calculadas.}
\label{table:ej1}
\end{table}

A continuación se grafica, para cada circuito, un diagrama fasorial de la corriente y la tensión:

\begin{figure}[H]
\centering
\begin{minipage}{.5\textwidth}
	\centering
	\includegraphics[width=1.2\linewidth]{Fasorial-1A.png}
	\caption{Fasores del circuito A.}
	\label{fig:faso-1a}
\end{minipage}
\begin{minipage}{.5\textwidth}
	\centering
	\includegraphics[width=1.2\linewidth]{Fasorial-1B.png}
	\caption{Fasores del circuito B.}
	\label{fig:faso-1b}
\end{minipage}\\
\end{figure}

\begin{figure}[H]
\centering
\begin{minipage}{.5\textwidth}
	\centering
	\includegraphics[width=1.2\linewidth]{Fasorial-1C.png}
	\caption{Fasores del circuito C.}
	\label{fig:faso-1c}
\end{minipage}
\end{figure}

Analizando los resultados obtenidos se pueden destacar las siguientes situaciones:
\begin{enumerate}

	\item Al dejar enteramente el núcleo sólido en la bobina, la potencia activa baja, ya que la inductancia en la bobina aumenta y ello genera una menor circulación de corriente por la resistencia. En este caso, se nota la menor potencia activa de los 3 casos.

	\item Al retirar el núcleo metálico, la inductancia en la bobina baja, por consiguiente aumenta la potencia activa del circuito, de forma que la circulación de corriente por la resistencia ahora es mayor. En este caso se nota que el crecimiento de la potencia aparente es un mucho mayor que del caso A al B donde hay un bajo aumento de la potencia activa, debido a que el efecto de tener mitad de núcleo o núcleo entero en la bobina, no genera una variación significativa en la inductancia, por ende en la corriente que circula por la resistencia.

	\item Cabe destacar que el factor de potencia se mantiene prácticamente inalterado en los casos A y B. Se atribuye dicho error a la poca diferencia entre la resistencia y la reactancia.

	\item Con los diagramas fasoriales de corrientes y tensiones, se puede ver la variación de intensidad de corrientes entre los casos A y B y el mayor crecimiento hacia el caso C, dejando como consecuencia lo descripto previamente.

\end{enumerate}

\subsection*{\underline{Ejercicio 2}}

En la segunda etapa se medió la potencia activa, reactiva y aparente, junto al valor del $cos(\varphi)$ de tres circuitos diferentes: R, C y RLC.

\begin{figure}[H]
\centering
\begin{minipage}{.5\textwidth}
  \centering
  \includegraphics[width=.8\linewidth]{Circuito-ejercicio-2A}
  \caption{Circuito R}
\end{minipage}%\\
\begin{minipage}{.5\textwidth}
  \centering
  \includegraphics[width=.8\linewidth]{Circuito-ejercicio-2B}
  \caption{Circuito C}

\end{minipage}%\\
\end{figure}
\begin{figure}[H]
\centering
\begin{minipage}{.5\textwidth}
  \centering
  \includegraphics[width=.8\linewidth]{Circuito-ejercicio-2C}
  \caption{Circuito RLC}
  \label{fig:2b}
\end{minipage}
\end{figure}

Obteniendo, los siguientes valores:

\begin{table}[H]
\centering
\begin{tabular}{|c|c|c|c|c|}
\hline
\textbf{Circuito} & \textbf{P {[}W{]}} & \textbf{Q {[}VAR{]}} & \textbf{S {[}VA{]}} & \textbf{Cos($\varphi$)} \\ \hline
R                 & 54,5               & 6,23                 & 54,86                & 0,99                                \\ \hline
C                 & 9,38               & 106,09               & 106,5                & 0,09                                \\ \hline
RLC               &  86              & 78,01                & 116.11                  & 0,74                                \\ \hline
\end{tabular}
\caption{Valores medidos de potencia.}
\end{table}

Graficando fasorialmente las tres potencias, se obtienen:

\begin{figure}[H]
\centering
\begin{minipage}{.5\textwidth}
  \centering
  \includegraphics[width=1.2\linewidth]{Diag-Fas-2A}
  \caption{Circuito R.}
\end{minipage}%\\
\begin{minipage}{.5\textwidth}
  \centering
  \includegraphics[width=1.2\linewidth]{Diag-Fas-2B}
  \caption{Circuito C.}

\end{minipage}\\
\begin{minipage}{.5\textwidth}
  \centering
  \includegraphics[width=1.2\linewidth]{Diag-Fas-2C}
  \caption{Circuito RLC.}
\end{minipage}
\end{figure}
 Se puede destacar que, por un lado, el factor de potencia es de aproximadamente 1 para un circuito puramente resistivo (intensidad y tensión en fase). Por otro, el factor de potencia es aproximadamente 0 para un circuito puramente reactivo (intensidad y tensión con una diferencia de 90 grados).
 
 De los gráficos de potencia y los datos medidos, se puede confirmar, según el circuito analizado, los comportamientos esperados: 
\begin{enumerate}

	\item En el circuito puramente resistivo, se observa que la potencia aparente es prácticamente la potencia activa, ya que no hay componentes reactivos, por ende no hay potencia reactiva. 
 
	\item En el circuito puramente reactivo (capacitivo), se denota un factor de potencia tal que el triangulo de potencias se encuentra en el cuarto cuadrante correspondiente a un triángulo de potencias teórico similar para este caso donde la potencia aparente es aproximadamente la potencia reactiva.
 
	\item En el circuito RLC, se encuentra el triángulo de potencias tal que el efecto inductivo es mayor al capacitivo, por lo cual se ubica en el primer cuadrante dando como resultado un circuito con comportamiento de forma inductiva. 
	
\subsection*{\underline{Ejercicio 3}}
La finalidad de la tercer etapa del trabajo consistió en variar el factor de potencia del circuito de la figura (\ref{fig:2b}) a un valor de $ 0.9 $ mediante la adición de capacitores en paralelo. En un principio se agregó un total de $ 16 \ \mu F$, lo que implicó que $ cos \left(\varphi \right) = 0.74 $. Al subir la capacitancia en paralelo a $ 24 \ \mu F$, se observo que $ cos \left(\varphi \right) = 0.91 $.
\end{enumerate}

\subsection*{\underline{Ejercicio 4}}
En la sección final de este trabajo, se buscó determinar analíticamente el valor capacitivo del banco de capacitores que se debería colocar en paralelo a los circuitos del ejercicio 2 para alcanzar un factor de potencia de $ 0,9 $. 
\begin{enumerate}

	\item Para el circuito de R, no tiene sentido colocar en paralelo un capacitor para subir el factor de potencia a $ 0,9 $ ya que el de dicho circuito es de $ 0,99 $.

	\item Para el circuito de C no tiene sentido agregar un capacitor para arreglar el cos($\varphi$) dado que se necesitaría una inductancia y el circuito presentado es de caracter capacitivo puro. 

	\item Por último, para el circuito RLC (que coincide con el circuito de la figura 9) se requiere un banco de $ 6,54 \ \mu F $. Este valor discrepa ligeramente con el obtenido experimentalmente debido a que en el modelo no se tomó en cuenta la impedancia de la fuente conjunto a errores de medición. Más allá de éstos, se observa que los cálculos teóricos se ajustan a lo medido.
\end{enumerate}

\end{document}
