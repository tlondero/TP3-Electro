\documentclass[a4paper]{article}
\usepackage[utf8]{inputenc}
\usepackage[spanish, es-tabla]{babel}

\usepackage{amsmath}
\usepackage{amsfonts}
\usepackage{amssymb}

\usepackage{float}
\usepackage{graphicx}
\graphicspath{ {./Imagenes/} }

\usepackage{multirow}
\setlength{\doublerulesep}{\arrayrulewidth}

\usepackage[american]{circuitikz}

\usepackage{fancyhdr}

\usepackage{units} 

\pagestyle{fancy}
\fancyhf{}
\lhead{22.02 Electrotecnia I}
\rhead{Mechoulam, Mestanza, Lambertucci, Pouthier, Londero}
\rfoot{Página \thepage}



\begin{document}

%%%%%%%%%%%%%%%%%%%%%%%%%%%%%%%%%%%%%%%%%%%%%%%%%%%%%%%%%%%%%%%%%%%%%%%%% 
%								CARATULA								%
%%%%%%%%%%%%%%%%%%%%%%%%%%%%%%%%%%%%%%%%%%%%%%%%%%%%%%%%%%%%%%%%%%%%%%%%% 

\begin{titlepage}
\newcommand{\HRule}{\rule{\linewidth}{0.5mm}}
\center
\mbox{\textsc{\LARGE \bfseries {Instituto Tecnológico de Buenos Aires}}}\\[1.5cm]
\textsc{\Large 22.02 Electrotecnia I}\\[0.5cm]


\HRule \\[0.6cm]
{ \Huge \bfseries Trabajo Práctico Final}\\[0.4cm] 
\HRule \\[1.5cm]


{\large

\emph{Grupo 5}\\
\vspace{3px}

\begin{tabular}{lr} 	
\textsc{Mechoulam}, Alan  &  58438\\
\textsc{Lambertucci}, Guido Enrique  & 58009 \\
\textsc{Pouthier}, Florian  & 61337 \\
\textsc{Mestanza}, Nicolás  & 57521 \\
\textsc{Londero Bonaparte}, Tomás Guillermo  & 58150 \\
\end{tabular}

\vspace{20px}

\emph{Profesores}\\
\vspace{3px}
\textsc{Muñoz}, Claudio Marcelo\\ 	
\textsc{Ayub}, Gustavo\\ 	

\vspace{100px}

\begin{tabular}{ll}

Presentado: & ??/06/19\\

\end{tabular}

}

\vfill

\end{titlepage}


%%%%%%%%%%%%%%%%%%%%%%%%%%%%%%%%%%%%%%%%%%%%%%%%%%%%%%%%%%%%%%%%%%%%%%%%% 
%								INFORME									%
%%%%%%%%%%%%%%%%%%%%%%%%%%%%%%%%%%%%%%%%%%%%%%%%%%%%%%%%%%%%%%%%%%%%%%%%%

\section*{Introducción}

El objetivo del trabajo consistió en el estudio del funcionamiento de un transformador. Se realizaron análisis sobre sistemas en corto, en vacío y con carga. 


\section*{Desarrollo de la experiencia}

\section[I]{\underline{Primera parte}}

Para empezar, se armó el siguiente circuito:

\begin{figure}[H]
\begin{center}
\begin{circuitikz}
	\draw
		
	(-1, -0.5) 		to [vL] (-1,-2.1)
	(-1.5, -0.5) 		to [short, *-] (-1, -0.5)
	(-1.5, -2.1) 	to [short, *-] (-1, -2.1)
					to (0,-2.1)
	(0,0)	to (0,-1)
			to (-0.6,-1)
	(0,0) 	to node[draw,circle,fill=white] {A} (1.5, 0)
			to (2.5,0) to [R, l=$R_{L_{1}}$] (4,0)
	(2,0)	to node[draw,circle,fill=white] {V} (2, -2.1)
	(0,-2.1)	to (4,-2.1)	
	(5,0)	node[transformer core]{}
	(7.5,0) 	to node[draw,circle,fill=white] {A} (10, 0)
	
	%(4,-2.1) to (5.5,-2.1)			%creo que Alan quería poner esta linea, a mi me aprece bien así como está
	
	(11.5,0)	to node[draw,circle,fill=white] {V} (11.5, -2.1)
	(10,0)	to (11.5,0)
	(10,-2.1)	to (11.5,-2.1)	
	(6,0)	to [R, l=$R_{L_{2}}$] (7.5,0)
	(6,-2.1)	to (10,-2.1)
	(10,0)	to [vR, l_ = $R_d$] (10,-2.1)

	;\end{circuitikz}
\end{center}
\caption{Circuito armado.}
\label{cir:1}
\end{figure}

Luego, antes de empezar las mediciones, se quiso comprobar la polaridad de las inductancias acopladas. Para esto, se conectó una fuente de tensión en el primario y un voltímetro entre los distintos bornes de las inductancias. Se identificaron los \textit{puntos} de las inductancias en los bornes donde el voltímetro medía la menor diferencia de potencial (aproximadamente 8V para el caso de mayor diferencia de potencial, y 3V para el de menor), de la siguiente manera:

\begin{figure}[H]
\begin{center}
\begin{circuitikz}
	\draw
		
	(1,0) node[transformer] (T) {}
	node[ocirc] (A) at ([xshift=-1cm]T.A1) {}
	node[ocirc] (B) at ([xshift=-1cm]T.A2) {}
	node[ocirc] (C) at ([xshift=1cm]T.B1) {}
	node[ocirc] (D) at ([xshift=1cm]T.B2) {}
	node[circ] (E) at ([xshift=0.4cm,yshift=-5pt]T.A1) {}
	node[circ] (F) at ([xshift=-0.4cm,yshift=-5pt]T.B1) {}
	(T.A1) to	[-o] (A)
	(T.A2) to	[-o] (B) 
	(T.B1) to	[-o] (C)
	(T.B2) to	[-o] (D)
	(T.west) node{$L_1$}
	(T.east) node{$L_2$}

	(0,0.5)	to	node[draw,circle,fill=white] {V} (2,0.5)
	(0,0)	to	(0,0.5)
	(2,0)	to	(2,0.5)

	([xshift=-3cm]T.A1) 		to [vL] ([xshift=-3cm,]T.A2)
	([xshift=-4cm]T.A2) 		to [short, *-] ([xshift=-3cm,]T.A2)
	([xshift=-4cm]T.A1) 		to [short, *-] ([xshift=-3cm,]T.A1)

	([xshift=-4cm]T.A2) 		to (T.A2)
	
	([xshift=-2cm]T.A1) 		to (T.A1)
	([xshift=-2cm]T.A1)			to ([xshift=-2cm,yshift=-1cm]T.A1)
	([xshift=-2cm,yshift=-1cm]T.A1) to ([xshift=-2.5cm,yshift=-1cm]T.A1)

	;\end{circuitikz}
\end{center}
\caption{Determinación de la polaridad}
\label{cir:pol}
\end{figure}

Se utilizó un ohmetro para medir la resistencia del devanado de cobre de los inductores, dando así $ R_1 = 23 \ \Omega $ y $ R_2 = 21,8 \ \Omega$ y finalmente se procedió a realizar las siguientes mediciones:

\begin{table}[H]
\centering
\begin{tabular}{|c|c|c|c|c|c|c|c|c|c|}
\hline
\textbf{Caso}                                                             & \textbf{\begin{tabular}[c]{@{}c@{}}$V_1$\\ {[}V{]}\end{tabular}} & \textbf{\begin{tabular}[c]{@{}c@{}}$I_1$\\ {[}A{]}\end{tabular}} & \textbf{\begin{tabular}[c]{@{}c@{}}$V_2$\\ {[}V{]}\end{tabular}} & \textbf{\begin{tabular}[c]{@{}c@{}}$I_2$\\ {[}A{]}\end{tabular}} & \textbf{\begin{tabular}[c]{@{}c@{}}$R_D$\\ $\Omega$\end{tabular}} & \textbf{\begin{tabular}[c]{@{}c@{}}M\\ {[}H{]}\end{tabular}} & \textbf{k} & \textbf{\begin{tabular}[c]{@{}c@{}}$L_1$\\ {[}H{]}\end{tabular}} & \textbf{\begin{tabular}[c]{@{}c@{}}$L_2$\\ {[}H{]}\end{tabular}} \\ \hline
\textbf{Hierro Sólido}                                                    & 93,4                                                             & 0,3                                                              & 14,6                                                             & 0,06                                                             & 200                                                               & 0,15                                                         & 0,28       & 0,99                                                             & 0,32                                                             \\ \hline
\textbf{Laminado}                                                         & 93,4                                                             & 0,45                                                             & 23,22                                                            & 0,1                                                              & 200                                                               & 0,26                                                         & 0,33       & 0,66                                                             & 0,93                                                             \\ \hline
\textbf{\begin{tabular}[c]{@{}c@{}}Laminado\\ ($I_2 = 0$)\end{tabular}}   & 93,4                                                             & 0,4                                                              & 33,22                                                            & 0                                                                & 0                                                                 & -                                                            & -          & -                                                                & -                                                                \\ \hline
\textbf{\begin{tabular}[c]{@{}c@{}}Sin núcleo\\ ($I_2 = 0$)\end{tabular}} & 93,4                                                             & 0,8                                                              & 8,3                                                              & 0                                                                & 0                                                                 & -                                                            & -          & -                                                                & -                                                                \\ \hline
\end{tabular}
\caption{Mediciones realizadas en la primer experiencia.}
\label{tabla:1a}
\end{table}

\section{\underline{Segunda parte}}

En esta parte, se hicieron estudios en vacío y de corto circuito en un transformador monofásico con el objetivo que hallar los parámetros físicos de este. El dicho es construido de la misma manera que en la primera parte, con dos bobinas y un núcleo de hierro.

No se realizó un estudio del transformador con carga ya que, como este era un modelo experimental, no se poseía la información de la potencia máxima.

\subsection{Ensayo en vacío}

Para este primer ensayo, se procede al armado del siguiente circuito:

\begin{figure}[H]
\begin{circuitikz}
\draw
	(-1, -0.5) 		to [vL] (-1,-2.1)
	(-1.5, -0.5) 		to [short, *-] (-1, -0.5)
	(-1.5, -2.1) 	to [short, *-] (-1, -2.1)
					to (0,-2.1)
	(0,0)	to (0,-1)
			to (-0.6,-1)
	(0,0) 	to node[draw,circle,fill=white] {A} (1.5, 0)
			to node[draw,circle,fill=white] {W} (3.5, 0)
			to [short, -*] (5, 0) to (5.5,0)
	(0.7,0.7) node[]{$I_{10}$}
	(2.5,0.7) node[]{$P_{10}$}
	(2.5,-0.4) to (2.5,-2.1)
	(4,0) to node[draw,circle,fill=white] {V} (4, -2.1)
	(4.5,-0.5) node[]{$U_{10}$}
	(0,-2.1) to [short, -*] (5,-2.1) to (5.5,-2.1)
	(6.5,0) node[transformer core]{}
	(8,0) to (7,0) to [short, -*] (9.5,0)
	(8,-2.1) to (7,-2.1) to [short, -*] (9.5,-2.1)
	(8.5,0) to node[draw,circle,fill=white] {V} (8.5, -2.1)
	(9,-0.5) node[]{$U_{20}$};
\end{circuitikz}
\caption{Circuito analizado durante la segunda instancia.}
\end{figure}

Aplicando una tensión nominal cercana de $100 \ V$, el vatímetro indicará las pérdidas en el hierro nominales. A continuación, se representa en forma de tabla, los valores medidos y calculados de dicho análisis:

\begin{table}[H]
\centering
\begin{tabular}{|c|c|c|c|c|c|c|c|c|c|c|}
\hline
\textbf{Parámetro} & \textbf{\begin{tabular}[c]{@{}c@{}}$U_{10}$\\ {[}V{]}\end{tabular}} & \textbf{\begin{tabular}[c]{@{}c@{}}$U_{20}$\\ {[}V{]}\end{tabular}} & \textbf{\begin{tabular}[c]{@{}c@{}}$I_{10}$\\ {[}A{]}\end{tabular}} & \textbf{\begin{tabular}[c]{@{}c@{}}$P_{10}$\\ {[}W{]}\end{tabular}} & \textbf{$cos \ \phi$} & \textbf{\begin{tabular}[c]{@{}c@{}}$I_{m}$\\ {[}A{]}\end{tabular}} & \textbf{\begin{tabular}[c]{@{}c@{}}$I_{p}$\\ {[}A{]}\end{tabular}} & \textbf{\begin{tabular}[c]{@{}c@{}}$R_{p}$\\ {[}$\Omega${]}\end{tabular}} & \textbf{\begin{tabular}[c]{@{}c@{}}$X_{m}$\\ {[}$\Omega${]}\end{tabular}} & \textbf{\begin{tabular}[c]{@{}c@{}}$M$\\ {[}H{]}\end{tabular}} \\ \hline
\textbf{Valor}     & 93,1                                                                & 35,2                                                                & 0,375                                                               & 6,75                                                                & 0,193                 & 0,368                                                              & 0,073                                                              & 1284,09                                                                   & 253,041                                                                   & 0,378                                                          \\ \hline
\end{tabular}
\caption {Ensayo con circuito abierto.}
\end{table}

\subsection{Ensayo en cortocircuito}
Se buscó determinar las pérdidas debidas al devanado de cobre del transformador. Para esto, se conectó un vatímetro en el primario y se concretó que la medición realizada con este serán las perdidas nominales en el cobre. A continuación se detallan los resultados obtenidos:

\begin{table}[H]
\centering
\begin{tabular}{|c|c|c|c|c|c|c|c|}
\hline
\textbf{Parámetro} & \textbf{\begin{tabular}[c]{@{}c@{}}$U_{1CC}$\\ {[}V{]}\end{tabular}} & \textbf{\begin{tabular}[c]{@{}c@{}}$I_{1CC}$\\ {[}A{]}\end{tabular}} & \textbf{\begin{tabular}[c]{@{}c@{}}$I_{2CC}$\\ {[}A{]}\end{tabular}} & \textbf{\begin{tabular}[c]{@{}c@{}}$P_{1CC}$\\ {[}W{]}\end{tabular}} & \textbf{$cos \ \phi$} & \textbf{\begin{tabular}[c]{@{}c@{}}$R_{1} = R_{21}$\\ {[}$\Omega${]}\end{tabular}} & \textbf{\begin{tabular}[c]{@{}c@{}}$X_{1} = X_{21}$\\ {[}$\Omega${]}\end{tabular}} \\ \hline
\textbf{Valor}     & 91,4                                                                 & 0,448                                                                & 0,16                                                                 & 8,625                                                                & 0,211                 & 336,9                                                                              & 1561,79                                                                            \\ \hline
\end{tabular}
\caption {Ensayo en corto circuito}
\centering
\end{table}

\section*{Conclusiones}
El trabajo permitió constatar la teoría estudiada del funcionamiento de los transformadores, viendo sus contraparte práctica y los efectos del acoplamiento magnético. 

Se pudo analizar el funcionamiento de un transformador y  la determinación de sus características más importantes, mediante diversos ensayos, observando variaciones debido a la utilización de elementos con una mayor inducción electromagnética relativa entre los diferentes experimentos para la primera parte.

En la segunda parte de la experiencia, en el primer ensayo realizado se pudo determinar la impedancia de vacío la cual representa tanto la inductancia de magnetización del núcleo como las pérdidas en el hierro nominales. En el segundo ensayo la impedancia de cortocircuito representó las pérdidas en el cobre de los devanados, así como la inductancia de dispersión y otras inductancias parásitas y utilizando el vatímetro se pudo obtener el valor nominal de dichas perdidas.

Los resultados obtenidos en cada uno de los experimentos corresponden con lo esperado, pero se ha tenido en cuenta la existencia de posibles errores debido a los instrumentos utilizados en las mediciones y efectos asociados a los elementos utilizados.

\end{document}
