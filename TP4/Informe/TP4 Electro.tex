\documentclass[a4paper]{article}
\usepackage[utf8]{inputenc}
\usepackage[spanish, es-tabla]{babel}

\usepackage{amsmath}
\usepackage{amsfonts}
\usepackage{amssymb}

\usepackage{float}
\usepackage{graphicx}
\graphicspath{ {./Imagenes/} }

\usepackage{multirow}
\setlength{\doublerulesep}{\arrayrulewidth}

\usepackage[american]{circuitikz}

\usepackage{fancyhdr}

\usepackage{units} 

\pagestyle{fancy}
\fancyhf{}
\lhead{22.02 Electrotecnia I}
\rhead{Mechoulam, Mestanza, Lambertucci, Pouthier, Londero}
\rfoot{Página \thepage}



\begin{document}

%%%%%%%%%%%%%%%%%%%%%%%%%%%%%%%%%%%%%%%%%%%%%%%%%%%%%%%%%%%%%%%%%%%%%%%%% 
%								CARATULA								%
%%%%%%%%%%%%%%%%%%%%%%%%%%%%%%%%%%%%%%%%%%%%%%%%%%%%%%%%%%%%%%%%%%%%%%%%% 

\begin{titlepage}
\newcommand{\HRule}{\rule{\linewidth}{0.5mm}}
\center
\mbox{\textsc{\LARGE \bfseries {Instituto Tecnológico de Buenos Aires}}}\\[1.5cm]
\textsc{\Large 22.02 Electrotecnia I}\\[0.5cm]


\HRule \\[0.6cm]
{ \Huge \bfseries Trabajo Práctico Final}\\[0.4cm] 
\HRule \\[1.5cm]


{\large

\emph{Grupo 5}\\
\vspace{3px}

\begin{tabular}{lr} 	
\textsc{Mechoulam}, Alan  &  58438\\
\textsc{Lambertucci}, Guido Enrique  & 58009 \\
\textsc{Pouthier}, Florian  & 61337 \\
\textsc{Mestanza}, Nicolás  & 57521 \\
\textsc{Londero Bonaparte}, Tomás Guillermo  & 58150 \\
\end{tabular}

\vspace{20px}

\emph{Profesores}\\
\vspace{3px}
\textsc{Muñoz}, Claudio Marcelo\\ 	
\textsc{Ayub}, Gustavo\\ 	

\vspace{100px}

\begin{tabular}{ll}

Presentado: & ??/06/19\\

\end{tabular}

}

\vfill

\end{titlepage}


%%%%%%%%%%%%%%%%%%%%%%%%%%%%%%%%%%%%%%%%%%%%%%%%%%%%%%%%%%%%%%%%%%%%%%%%% 
%								INFORME									%
%%%%%%%%%%%%%%%%%%%%%%%%%%%%%%%%%%%%%%%%%%%%%%%%%%%%%%%%%%%%%%%%%%%%%%%%%

\section*{Introducción}

En el trabajo realizado se analizó el funcionamiento de un transformador. Se realizaron análisis en del circuito analizando las situaciones en corto, en vacío y con carga. 
%Para llevar adelante el experimento se utilizó:
%\begin{itemize}
%	\item[a)]	Ohmetro;
%	\item[b)]	Voltímetro;
%	\item[c)]	Amperímetro;
%	\item[d)]	Vatímetro;
%	\item[e)]	Inductancias; y
%	\item[f)]	Autotransformador.
%\end{itemize}

\section*{Desarrollo de la experiencia}

\section[I]{\underline{Primera parte}}

Para empezar se dispuso el circuito de la forma mostrada en la figura (\ref{fig:1a}). Luego, se colocó un voltímetro en los distintos bornes de las inductacias, de esta forma se buscó determinar de que manera serían homólogos, sabiendo que se daría dicha condición cuando la diferencia de tensión sea próxima a cero.

\begin{figure}[H]
	\centering
	\includegraphics[width=0.8\textwidth]{Circuito-1.PNG}
	\caption{Circuito empleado.}
	\label{fig:1a}
\end{figure}

Así queda determinado el sentido de la inductancia mutua, conectando el circuito de forma tal que se vea de la forma mostrada en la figura (\ref{cir:1a}).

\begin{figure}[H]
\begin{center}
\begin{circuitikz}
	\draw
		
	(1,0) node[transformer] (T) {}
	node[ocirc] (A) at ([xshift=-1cm]T.A1) {}
	node[ocirc] (B) at ([xshift=-1cm]T.A2) {}
	node[ocirc] (C) at ([xshift=1cm]T.B1) {}
	node[ocirc] (D) at ([xshift=1cm]T.B2) {}
	node[circ] (E) at ([xshift=0.4cm,yshift=-5pt]T.A1) {}
	node[circ] (F) at ([xshift=-0.4cm,yshift=-5pt]T.B1) {}
	(T.A1) to	[-o] (A)
	(T.A2) to	[-o] (B) 
	(T.B1) to	[-o] (C)
	(T.B2) to	[-o] (D)
	(T.west) node{$L_1$}
	(T.east) node{$L_2$}

	(0,0.5)	to	node[draw,circle,fill=white] {V} (2,0.5)
	(0,0)	to	(0,0.5)
	(2,0)	to	(2,0.5)

	;\end{circuitikz}
\end{center}
\caption{Determinación del sentido de M.}
\label{cir:1a}
\end{figure}

Una vez hecho lo mencionado anteriormente, se prosigue conectando las herramientas necesarias en el circuito para realizar las mediciones adecuadas.

\begin{figure}[H]
\begin{center}
\begin{circuitikz}
	\draw
		
	(-1, -1.1) 		to [vL] (-1,-2.1)
	(-1.5, -1.1) 	to [short, *-] (-1, -1.1)
	(-1.5, -2.1) 	to [short, *-] (-1, -2.1)
					to (0,-2.1)
	(0,0)	to (0,-1.7)
			to (-0.6,-1.7)
	(0,0) 	to node[draw,circle,fill=white] {A} (1.5, 0)
			to (3.5,0) to [R, l=$R_{L_{1}}$] (5.5,0)
			
	(2.5,0) to node[draw,circle,fill=white] {V} (2.5, -2.1)
	
	(0,-2.1) to (5,-2.1) to (5.5,-2.1)	

	(6.5,0) node[transformer core]{}
	(7.5,0) to [R, l=$R_{L_{2}}$] (9.5,0) to (10,0)
	(7.5,-2.1) to (10,-2.1)
	(10,0) to [vR, l_ = $R_d$] (10,-2.1)

	;\end{circuitikz}
\end{center}
\caption{Circuito empleado con las conexiones necesarias.}
\label{cir:1b}
\end{figure}

Luego se determinaron las resisetncias internas de cada bobina siendo $ R_1 = 23 \ \Omega $ y $ R_2 = 21,8 \ \Omega $ las resistencias del primario y del secundario respectivamente.

Por último se procedió a realizar las mediciones adecuadas y se calcularon los valores de auto inductancia, el factor de acoplamiento y el valor de mutua inductancia. Dichos resultados y cálculos se encuentran en la tabla (\ref{tabla:1a}). Cabe aclarar que algunos valores no pudieron ser calculados dada las condiciones, como por ejemplo en los análisis en vacío. 

\begin{table}[H]
\begin{tabular}{|c|c|c|c|c|c|c|c|c|c|}
\hline
\textbf{Caso} & $V_1 \ [V]$ & $I_1 \ [A]$ & $V_2 \ [V]$ & $I_2 \ [A]$ & $R_D \ [\Omega]$ & M [H]& k & $L_1 [H]$ & $L_2 [H]$ \\ \hline
\textbf{Hierro Sólido} & 93,4 & 0,3 & 14,6 & 0,06 & 200 & 0,15 & 0,28 & 0,99 & 0,32 \\ \hline
\textbf{Laminado} & 93,4 & 0,45 & 23,22 & 0,1 & 200 & 0,26 & 0,33 & 0,66 & 0,93 \\ \hline
\textbf{Laminado ($I_2 = 0$)} & 93,4 & 0,4 & 33,22 & 0 & 0 & 0,26 & - & - & - \\ \hline
\textbf{Sin núcleo ($I_2 = 0$)} & 93,4 & 0,8 & 8,3 & 0 & 0 & 0,3 & - & - & - \\ \hline
\end{tabular}
\caption{Mediciones realizadas en la primer experiencia.}
\label{tabla:1a}
\end{table}

\section{\underline{Segunda parte}}

En esta parte, se va a hacer un análisis práctico de un transformador monofásico. El dicho es construido de la misma manera que antes con dos bobinas y un núcleo de hierro.

El objetivo es entonces de hallar los parámetros físicos de este transformador con diferentes ensayos sucesivos.

Con uno transformador de ese tipo, no se puede estimar rendimiento o regulación, porque ...

\subsection{Ensayo en vacío}

Para ese primer ensayo, se procede al armado del siguiente circuito :

\begin{circuitikz}
\draw
	(-1, -1.1) 		to [vL] (-1,-2.1)
	(-1.5, -1.1) 	to [short, *-] (-1, -1.1)
	(-1.5, -2.1) 	to [short, *-] (-1, -2.1)
					to (0,-2.1)
	(0,0)	to (0,-1.7)
			to (-0.6,-1.7)
	(0,0) 	to node[draw,circle,fill=white] {A} (1.5, 0)
			to node[draw,circle,fill=white] {W} (3.5, 0)
			to [short, -*] (5, 0) to (5.5,0)
	(0.7,0.7) node[]{$I_{10}$}
	(2.5,0.7) node[]{$P_{10}$}
	(2.5,-0.4) to (2.5,-2.1)
	(4,0) to node[draw,circle,fill=white] {V} (4, -2.1)
	(4.5,-0.5) node[]{$U_{10}$}
	(0,-2.1) to [short, -*] (5,-2.1) to (5.5,-2.1)
	(6.5,0) node[transformer core]{}
	(8,0) to (7,0) to [short, -*] (9.5,0)
	(8,-2.1) to (7,-2.1) to [short, -*] (9.5,-2.1)
	(8.5,0) to node[draw,circle,fill=white] {V} (8.5, -2.1)
	(9,-0.5) node[]{$U_{20}$};
\end{circuitikz}

Aplicando una tensión nominal cercana de 100 V, el vatímetro indicará las pérdidas en el hierro nominales.

\begin{table}[H]
\centering
\begin{tabular}{|l|l|l|l|l|l|l|l|l|l|l|}
\hline
Parámetro & $U_{10} [V]$ & $U_{20} [V]$ & $I_{10} [A]$ & $P_{10} [W] $ & $ cos \phi_0 $ & $I_m [A] $ & $I_p [A] $ & $ R_p [\Omega]$ & $ X_m [\Omega] $ & M     \\ \hline
Valor     & 93.1  & 35.2  & 0.375 & 6.75  & 0.193        & 0.368 & 0.073 & 1284.090 & 253.041     & 0.378 \\ \hline
\end{tabular}
\caption {Ensayo a circuito abierto}
\end{table}

\begin{table}[H]
\centering
\begin{tabular}{||c||c|c|c|c|c|c||}
\hline\hline
\multicolumn{7}{c}{\textbf{VALORES CALCULADOS}}\\
\hline\hline
\multirow{2}{*}{\textbf{Parámetro}} 	& \multirow{2}{*}{cos $\phi_0$} & $I_{m}$ & $I_{p}$ & $R_{p}$ & $X_{m}$ & $m$ \\
										&  & [A] & [A] & [$\Omega$] & [$\Omega$] & --\\
\hline										
Valor     								& 0.193        & 0.368 & 0.073 & 1284.090 & 253.041     & 0.378   \\ 
\hline\hline
\end{tabular}
\caption {Ensayo a circuito abierto}
\end{table}

\subsection{Ensayo en cortocircuito}
Se quiso encontrar las pérdidas debidas al devanado de cobre del transformador. Para esto, se conectó un vatímetro en el primario y se concretó que la medición realizada con este son las pérdidas, ya que la única resistencia del circuito es la del cobre. A continuación se detallan los resultados obtenidos.
\begin{table}[H]
\centering
\begin{tabular}{|l|l|l|l|l|l|l|l|}
\hline
Parámetro & $U_1 \ [V]$ & $I_1 \ [A]$ & $I_2 \ [A]$ & $P_1 \ [W]$ & $cos \ \phi_0 $  & $R_1 = R_21 \ [\Omega] $ & $X_1 = X_{21} \ [\Omega] $\\ \hline
Valor     & 91.4     & 0.448    & 0.16     & 8.625    & 0.211            & 336.9               & 1561.79             \\ \hline
\end{tabular}
\caption {Ensayo a corto circuito}
\centering
\end{table}
\subsection{Modelo resultante del transformador}



\end{document}
