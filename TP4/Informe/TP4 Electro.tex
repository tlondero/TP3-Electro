\documentclass[a4paper]{article}
\usepackage[utf8]{inputenc}
\usepackage[spanish, es-tabla]{babel}

\usepackage{amsmath}
\usepackage{amsfonts}
\usepackage{amssymb}

\usepackage{float}
\usepackage{graphicx}
\graphicspath{ {./Imagenes/} }

\usepackage[american voltage]{circuitikz}

\usepackage{fancyhdr}

\usepackage{units} 

\pagestyle{fancy}
\fancyhf{}
\lhead{22.02 Electrotecnia I}
\rhead{Mechoulam, Mestanza, Lambertucci, Pouthier, Londero}
\rfoot{Página \thepage}



\begin{document}

%%%%%%%%%%%%%%%%%%%%%%%%%%%%%%%%%%%%%%%%%%%%%%%%%%%%%%%%%%%%%%%%%%%%%%%%% 
%								CARATULA								%
%%%%%%%%%%%%%%%%%%%%%%%%%%%%%%%%%%%%%%%%%%%%%%%%%%%%%%%%%%%%%%%%%%%%%%%%% 

\begin{titlepage}
\newcommand{\HRule}{\rule{\linewidth}{0.5mm}}
\center
\mbox{\textsc{\LARGE \bfseries {Instituto Tecnológico de Buenos Aires}}}\\[1.5cm]
\textsc{\Large 22.02 Electrotecnia I}\\[0.5cm]


\HRule \\[0.6cm]
{ \Huge \bfseries Trabajo Práctico Final}\\[0.4cm] 
\HRule \\[1.5cm]


{\large

\emph{Grupo 5}\\
\vspace{3px}

\begin{tabular}{lr} 	
\textsc{Mechoulam}, Alan  &  58438\\
\textsc{Lambertucci}, Guido Enrique  & 58009 \\
\textsc{Pouthier}, Florian  & 61337 \\
\textsc{Mestanza}, Nicolás  & 57521 \\
\textsc{Londero Bonaparte}, Tomás Guillermo  & 58150 \\
\end{tabular}

\vspace{20px}

\emph{Profesores}\\
\vspace{3px}
\textsc{Muñoz}, Claudio Marcelo\\ 	
\textsc{Ayub}, Gustavo\\ 	

\vspace{100px}

\begin{tabular}{ll}

Presentado: & ??/06/19\\

\end{tabular}

}

\vfill

\end{titlepage}


%%%%%%%%%%%%%%%%%%%%%%%%%%%%%%%%%%%%%%%%%%%%%%%%%%%%%%%%%%%%%%%%%%%%%%%%% 
%								INFORME									%
%%%%%%%%%%%%%%%%%%%%%%%%%%%%%%%%%%%%%%%%%%%%%%%%%%%%%%%%%%%%%%%%%%%%%%%%%

\section*{Introducción}

\section*{Desarrollo de la experiencia}

\section[I]{\underline{Primera parte}}

\section{\underline{Segunda parte}}

En esta parte, se va a hacer un análisis práctico de un transformador monofásico. El dicho es construido de la misma manera que antes con dos bobinas y un núcleo de hierro.

El objetivo es entonces de hallar los parámetros físicos de este transformador con diferentes ensayos sucesivos.

Con uno transformador de ese tipo, no se puede estimar rendimiento o regulación, porque ...

\subsection{Ensayo en vacío}

Para ese primer ensayo, se procede al armado del siguiente circuito :

\begin{circuitikz}
\draw
	(0,0) to [ammeter] (1.5,0) to (3,0) to [short, -*] (3.5,0) 	
	(4.5,0) node[transformer core]{}
	(2.5,0) to node[draw,circle,fill=white] {V} (2.5, -2.1)
	(5.5,0) to (7,0) to [short, -*] (7.5,0)
	(5.5, -2.1) to (7, -2.1) to [short, -*] (7.5, -2.1)
	(7,0) to node[draw,circle,fill=white] {V} (7, -2.1)
	(7.5,-0.5) node[]{$U_{20}$};
\end{circuitikz}

Aplicando una tensión nominal cercana de 100 V, el vatímetro indicará las pérdidas en el hierro nominales.

\subsection{Ensayo en cortocircuito}

\subsection{Modelo resultante del transformador}

\end{document}
