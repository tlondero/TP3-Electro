\documentclass[a4paper]{article}
\usepackage[utf8]{inputenc}
\usepackage[spanish, es-tabla]{babel}

\usepackage{amsmath}
\usepackage{amsfonts}
\usepackage{amssymb}

\usepackage{float}
\usepackage{graphicx}
\graphicspath{ {./Imagenes/} }

\usepackage{multirow}
\setlength{\doublerulesep}{\arrayrulewidth}

\usepackage{array}
\newcolumntype{C}[1]{>{\centering\let\newline\\\arraybackslash\hspace{0pt}}m{#1}}

\usepackage[american]{circuitikz}

\usepackage{fancyhdr}

\usepackage{units} 

\pagestyle{fancy}
\fancyhf{}
\lhead{22.02 Electrotecnia I}
\rhead{Mechoulam, Mestanza, Lambertucci, Pouthier, Londero}
\rfoot{Página \thepage}



\begin{document}

%%%%%%%%%%%%%%%%%%%%%%%%%%%%%%%%%%%%%%%%%%%%%%%%%%%%%%%%%%%%%%%%%%%%%%%%% 
%								CARATULA								%
%%%%%%%%%%%%%%%%%%%%%%%%%%%%%%%%%%%%%%%%%%%%%%%%%%%%%%%%%%%%%%%%%%%%%%%%% 

\begin{titlepage}
\newcommand{\HRule}{\rule{\linewidth}{0.5mm}}
\center
\mbox{\textsc{\LARGE \bfseries {Instituto Tecnológico de Buenos Aires}}}\\[1.5cm]
\textsc{\Large 22.02 Electrotecnia I}\\[0.5cm]


\HRule \\[0.6cm]
{ \Huge \bfseries Trabajo práctico N$^{\circ}$3}\\[0.4cm] 
\HRule \\[1.5cm]


{\large

\emph{Grupo 5}\\
\vspace{3px}

\begin{tabular}{lr} 	
\textsc{Mechoulam}, Alan  &  58438\\
\textsc{Lambertucci}, Guido Enrique  & 58009 \\
\textsc{Pouthier}, Florian  & 61337 \\
\textsc{Mestanza}, Nicolás  & 57521 \\
\textsc{Londero Bonaparte}, Tomás Guillermo  & 58150 \\
\end{tabular}

\vspace{20px}

\emph{Profesores}\\
\vspace{3px}
\textsc{Muñoz}, Claudio Marcelo\\ 	
\textsc{Ayub}, Gustavo\\ 	

\vspace{100px}

\begin{tabular}{ll}

Presentado: & 17/05/19\\

\end{tabular}

}

\vfill

\end{titlepage}


%%%%%%%%%%%%%%%%%%%%%%%%%%%%%%%%%%%%%%%%%%%%%%%%%%%%%%%%%%%%%%%%%%%%%%%%% 
%								INFORME									%
%%%%%%%%%%%%%%%%%%%%%%%%%%%%%%%%%%%%%%%%%%%%%%%%%%%%%%%%%%%%%%%%%%%%%%%%%

\section*{Introducción}

En ese trabajo práctico, se va a determinar los parámetros de \textbf{impedencia [Z]}

\section*{Desarrollo de la experiencia}

\begin{table}
\begin{center}
\begin{tabular}{|||C{1.8cm}|C{1.8cm}|C{1.8cm}|C{1.8cm}|C{1.8cm}|||}
\hline\hline 
\multicolumn{5}{|||l|||}{}\\[-7pt]
\multicolumn{5}{|||l|||}{\textbf{Cuadripolo A - Nro de serie : 9601}} \\ 
\multicolumn{5}{|||l|||}{}\\[-7pt]
\hline
&&&&\\[-7pt]
 & $V_{1}$ (V) & $V_{2}$ (V) & $I_{1}$ (mA) & $I_{2}$ (mA) \\
&&&&\\[-7pt]
\hline
&&&&\\[-7pt]
$V_{1}=0$ & 0 &   &   &   \\
&&&&\\[-7pt]
\hline
&&&&\\[-7pt]
$V_{2}=0$ &   & 0 &   &   \\
&&&&\\[-7pt]
\hline
&&&&\\[-7pt]
$I_{1}=0$ &   &   & 0 &   \\
&&&&\\[-7pt]
\hline
&&&&\\[-7pt]
$I_{2}=0$ &   &   &   & 0 \\[-7pt]
&&&&\\
\hline\hline
\end{tabular}
\caption{Linda tabla}
\end{center}
\end{table}

\begin{table}
\begin{center}
\begin{tabular}{|||C{1.8cm}|C{1.8cm}|C{1.8cm}|C{1.8cm}|C{1.8cm}|||}
\hline\hline 
\multicolumn{5}{|||l|||}{}\\[-7pt]
\multicolumn{5}{|||l|||}{\textbf{Cuadripolo B - Nro de serie : 9612}} \\ 
\multicolumn{5}{|||l|||}{}\\[-7pt]
\hline
&&&&\\[-7pt]
 & $V_{1}$ (V) & $V_{2}$ (V) & $I_{1}$ (mA) & $I_{2}$ (mA) \\
&&&&\\[-7pt]
\hline
&&&&\\[-7pt]
$V_{1}=0$ & 0 &   &   &   \\
&&&&\\[-7pt]
\hline
&&&&\\[-7pt]
$V_{2}=0$ &   & 0 &   &   \\
&&&&\\[-7pt]
\hline
&&&&\\[-7pt]
$I_{1}=0$ &   &   & 0 &   \\
&&&&\\[-7pt]
\hline
&&&&\\[-7pt]
$I_{2}=0$ &   &   &   & 0 \\[-7pt]
&&&&\\
\hline\hline
\end{tabular}
\caption{Linda tabla}
\end{center}
\end{table}

\begin{table}[h]
\begin{center}
\begin{tabular}{|||C{1.8cm}|C{1.8cm}|C{1.8cm}|C{1.8cm}|C{1.8cm}|||}
\hline\hline 
\multicolumn{5}{|||l|||}{}\\[-7pt]
\multicolumn{5}{|||l|||}{\textbf{Cuadripolo A y B en serie :}} \\ 
\multicolumn{5}{|||l|||}{}\\[-7pt]
\hline
&&&&\\[-7pt]
 & $V_{1}$ (V) & $V_{2}$ (V) & $I_{1}$ (mA) & $I_{2}$ (mA) \\
&&&&\\[-7pt]
\hline
&&&&\\[-7pt]
$I_{1}=0$ &   &   & 0 &   \\
&&&&\\[-7pt]
\hline
&&&&\\[-7pt]
$I_{2}=0$ &   &   &   & 0 \\[-7pt]
&&&&\\
\hline\hline
\end{tabular}
\caption{Linda tabla}
\end{center}
\end{table}

\begin{table}[h]
\begin{center}
\begin{tabular}{|||C{1.8cm}|C{1.8cm}|C{1.8cm}|C{1.8cm}|C{1.8cm}|||}
\hline\hline 
\multicolumn{5}{|||l|||}{}\\[-7pt]
\multicolumn{5}{|||l|||}{\textbf{Cuadripolo A y B en paralelo :}} \\ 
\multicolumn{5}{|||l|||}{}\\[-7pt]
\hline
&&&&\\[-7pt]
 & $V_{1}$ (V) & $V_{2}$ (V) & $I_{1}$ (mA) & $I_{2}$ (mA) \\
&&&&\\[-7pt]
\hline
&&&&\\[-7pt]
$V_{1}=0$ & 0 &   &   &   \\
&&&&\\[-7pt]
\hline
&&&&\\[-7pt]
$V_{2}=0$ &   & 0 &   &   \\[-7pt]
&&&&\\
\hline\hline
\end{tabular}
\caption{Linda tabla}
\end{center}
\end{table}

\begin{table}[h]
\begin{center}
\begin{tabular}{|||C{1.8cm}|C{1.8cm}|C{1.8cm}|C{1.8cm}|C{1.8cm}|||}
\hline\hline 
\multicolumn{5}{|||l|||}{}\\[-7pt]
\multicolumn{5}{|||l|||}{\textbf{Cuadripolo A y B en cascada (cascada A-B) :}} \\ 
\multicolumn{5}{|||l|||}{}\\[-7pt]
\hline
&&&&\\[-7pt]
 & $V_{1}$ (V) & $V_{2}$ (V) & $I_{1}$ (mA) & $I_{2}$ (mA) \\
&&&&\\[-7pt]
\hline
&&&&\\[-7pt]
$V_{2}=0$ &   & 0 &   &   \\
&&&&\\[-7pt]
\hline
&&&&\\[-7pt]
$I_{2}=0$ &   &   &   & 0 \\[-7pt]
&&&&\\
\hline\hline
\end{tabular}
\caption{Linda tabla}
\end{center}
\end{table}

\begin{table}[h]
\begin{center}
\begin{tabular}{|||C{1.8cm}|C{1.8cm}|C{1.8cm}|C{1.8cm}|C{1.8cm}|||}
\hline\hline 
\multicolumn{5}{|||l|||}{}\\[-7pt]
\multicolumn{5}{|||l|||}{\textbf{Cuadripolo B y A en cascada (cascada B-A) :}} \\ 
\multicolumn{5}{|||l|||}{}\\[-7pt]
\hline
&&&&\\[-7pt]
 & $V_{1}$ (V) & $V_{2}$ (V) & $I_{1}$ (mA) & $I_{2}$ (mA) \\
&&&&\\[-7pt]
\hline
&&&&\\[-7pt]
$V_{2}=0$ &   & 0 &   &   \\
&&&&\\[-7pt]
\hline
&&&&\\[-7pt]
$I_{2}=0$ &   &   &   & 0 \\[-7pt]
&&&&\\
\hline\hline
\end{tabular}
\caption{Linda tabla}
\end{center}
\end{table}

\subsection*{Cálculos}

A partir de los valores medidos antes, se puede calcular los parámetros correspondientes de cada cuadripolo ensayado.

\subsubsection*{Parámetros de impedancia Z (impedancia a circuito abierto)}

\begin{equation}
\text{Cuadripolo A :}\quad [Z_{A}] =
\begin{vmatrix}
	Z^{A}_{11} & Z^{A}_{12}\\
	Z^{A}_{21} & Z^{A}_{22}\\
\end{vmatrix}
\end{equation}

\begin{equation}
\text{Cuadripolo B :}\quad [Z_{B}] =
\begin{vmatrix}
	Z^{B}_{11} & Z^{B}_{12}\\
	Z^{B}_{21} & Z^{B}_{22}\\
\end{vmatrix}
\end{equation}

\begin{equation}
\text{Cuadripolos en serie :}\quad [Z_{serie}] =
\begin{vmatrix}
	Z^{AB}_{11} & Z^{AB}_{12}\\
	Z^{AB}_{21} & Z^{AB}_{22}\\
\end{vmatrix}
\end{equation}

\subsubsection*{Parámetros de admitancia Y (admitancia a cortocircuito)}

\begin{equation}
\text{Cuadripolo A :}\quad [Y_{A}] =
\begin{vmatrix}
	Y^{A}_{11} & Y^{A}_{12}\\
	Y^{A}_{21} & Y^{A}_{22}\\
\end{vmatrix}
\end{equation}

\begin{equation}
\text{Cuadripolo B :}\quad [Y_{B}] =
\begin{vmatrix}
	Y^{B}_{11} & Y^{B}_{12}\\
	Y^{B}_{21} & Y^{B}_{22}\\
\end{vmatrix}
\end{equation}

\begin{equation}
\text{Cuadripolos en paralelo :}\quad [Y_{paralelo}] =
\begin{vmatrix}
	Y^{AB}_{11} & Y^{AB}_{12}\\
	Y^{AB}_{21} & Y^{AB}_{22}\\
\end{vmatrix}
\end{equation}

\subsubsection*{Parámetros de transmisión T}

\begin{equation}
\text{Cuadripolo A :}\quad [T_{A}] =
\begin{vmatrix}
	T^{A}_{11} & T^{A}_{12}\\
	T^{A}_{21} & T^{A}_{22}\\
\end{vmatrix}
\end{equation}

\begin{equation}
\text{Cuadripolo B :}\quad [T_{B}] =
\begin{vmatrix}
	T^{B}_{11} & T^{B}_{12}\\
	T^{B}_{21} & T^{B}_{22}\\
\end{vmatrix}
\end{equation}

\begin{equation}
\text{Cascada A-B :}\quad [T_{A-B}] =
\begin{vmatrix}
	T^{A-B}_{11} & T^{A-B}_{12}\\
	T^{A-B}_{21} & T^{A-B}_{22}\\
\end{vmatrix}
\end{equation}

\begin{equation}
\text{Cascada B-A :}\quad [T_{B-A}] =
\begin{vmatrix}
	T^{B-A}_{11} & T^{B-A}_{12}\\
	T^{B-A}_{21} & T^{B-A}_{22}\\
\end{vmatrix}
\end{equation}

\subsection*{Equivalencia de parámetros de cuadripolos sueltos}

A continuación, se verificará la correspondencia entre los valores medidos y los valores calculados con diferentes fórmulas conocidas.

\subsubsection*{Equivalencia [Z] - [Y]}

Está basada en la fórmula : $[Z]=[Y]^{-1}$.

\begin{equation}
ert
\end{equation}

\subsubsection*{Equivalencia [Z] - [T]}

Está basada en la transformación : 

\subsection*{Equivalencia de parámetros de cuadripolos asociados}

\subsubsection*{Conexión serie}

\subsubsection*{Conexión paralelo}

\subsubsection*{Conexión cascada A-B}

\subsubsection*{Conexión cascada B-A}

\section*{Conclusión}

\end{document}
